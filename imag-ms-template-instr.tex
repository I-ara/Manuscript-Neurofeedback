\documentclass[]{imag-ms-template}

\title{
{\small Manuscript created with Template for MIT Journals} \\
{Down-regulating the food-Salience Network of chocolate lovers with real time fMRI neurofeedback}
}

\author {Iara de Almeida Ivo,$^{1,2,3\ast}$ Anne Roefs,$^{2}$ Bettina Sorger,$^{1}$ David E.J. Linden$^{3}$\\
{\small $^{1}$Department of Cognitive Neuroscience, Maastricht University,}\\
{\small $^{2}$Department of Clinical Psychological Science, Maastricht University,}\\
{\small $^{3}$FHML, Mental Health and Neuroscience Research Institute, Maastricht University}\\
{\small $^\ast$Correspondence:  i.dealmeidaivo@maastrichtuniversity.nl}
}

\addbibresource{overleaf_neurofeedback.bib}

\begin{document} 

\maketitle 

\keywords{neurofeedback, rt-fMRI, large-scale networks and mindset}

\newpage

\begin{abstract}
  \\ \textbf{Background:} Food cues strongly influence eating behaviour, increasing intake through environmental context, variety, advertisements, and social factors. As overeating, or excessive food intake can be a symptom for certain eating disorders, there is a need for interventions that mitigate the appetite-enhancing effects of food cues. Neurofeedback has successfully been used to modulate neural activity, but due to the dynamic nature of food reward processing, there is a lack of clear neuroanatomical targets or consistent methodology. This study investigates whether real-time fMRI (rt-fMRI) neurofeedback can down-regulate the core food-salience network - which includes the right inferior parietal Lobe, the left inferior frontal gyrus, the left middle frontal gyrus, and the left superior frontal gyrus -, and whether this modulation reduces real food craving. By shifting attention away from hedonic attributes such as taste, we aim to alter the motivational salience of food cues and attenuate craving responses. \textbf{Method:} Thirty adults with normal BMI and a self-proclaimed love of chocolate were recruited. Using rt-fMRI neurofeedback, participants were instructed to down-regulate activity in individually localised top 10\% most responsive voxels of the core food-salience network. The participants were split into two groups: one receiving true neurofeedback and the other receiving yoked (sham) neurofeedback from another participant. Success was defined as a reduction in baseline activity in the food-salience network.
  \\\textbf{Results:} [To be determined] \\\textbf{Discussion:} [To be determined]
\end{abstract}

\newpage

\section{Introduction}

When it comes to food, the western world can be described as an overconsumption-promoting environment, as people are continuously exposed to easily accessible, high-calorie, palatable foods (\cite{hillObesityOverviewEpidemic2005}). Such an environment fosters habitual overeating and promotes unhealthy eating behaviours at the population level. Overeating or regular excessive eating can then lead to obesity (\cite{hillObesityOverviewEpidemic2005}), and binging behaviour patterns   

However, not everyone is equally susceptible to this environment. 

Previous studies (\cite{frankortRewardActivitySatiated2012, franssenPowerMindAttentional2020, franssenEffectsMindsetHormonal2022, pimpiniMoreComplexYou2022, kochsItMatterPerspective2023}) suggest that a person’s current mindset, also called attentional focus, is a more important determinant of how the brain responds to food than relatively stagnant characteristics (such as body mass index). Consequently, the neural response to food cues does not robustly reflect food reward value (i.e., no significant difference between palatable vs. unpalatable), but only salience (i.e., attentional focus); hence the term salience network instead of reward network.

It has consistently been shown that food cues increase the likelihood of food intake, through external factors like: environmental context (\cite{jansenCuedOvereating2011b}), food variety (\cite{guerrieriInteractionImpulsivityVaried2008, remickInternalExternalModerators2009}), advertisements (\cite{harrisPrimingEffectsTelevision2009}), and intake of other people (\cite{hermansMimicryFoodIntake2012}). As cue-elicited eating leads to overeating and weight gain(\cite{jansenCuedOvereating2011b, frankortCravingStopsYou2014}), reducing the appetite-enhancing influence of food cues could be a key strategy in promoting healthy eating behaviour.


\textbf{paragraph under construction - references broken} ------------------------------------------------------------
\\
While previous studies have investigated cognitive training and inhibitory control strategies to regulate food cravings, these behavioural interventions often exhibit limited long-term efficacy and fail to engage specific neurobiological targets.\cite{pasquale}

In contrast, neurofeedback offers a promising approach to modulate brain activity directly; however, existing food-related neurofeedback studies face critical limitations. \cite{Ihssen, Bartholdy, Kohl, Schmidt&Martin, Blume, Leong, de Clerck, Pasquale}

These include small sample sizes, variability in methodological design, and inconsistent selection of neural targets—ranging from prefrontal regions to reward-related areas—hindering a clear understanding of which neural circuits can be effectively modulated to attenuate craving.

These challenges underscore the need for more targeted, reproducible approaches to identify and engage specific brain regions involved in craving regulation.
\\
-----------------------------------------------------------------------------------------------------------------------------------

To bridge this gap, we use real-time fMRI (rt-fMRI) neurofeedback to directly modulate the core food-salience network—comprising the right Inferior Parietal Lobe, left Inferior Frontal Gyrus, left Middle Frontal Gyrus, and left Superior Frontal Gyrus — which we previously identified as central to the attentional focus on food (\textbf{(\textit{insert our 1st study here})}). In this study, we provide real-time feedback on neural activity within these regions while participants viewed food stimuli, thus allowing them to voluntarily regulate their response. By shifting attention away from hedonic attributes like taste, we aim to alter the motivational salience of food and assess whether this modulation reduces cravings.

Specifically, our study determines whether the food-salience network activity can be downregulated to reduce chocolate craving. To achieve this, we implemented a structured rt-fMRI neurofeedback protocol designed to test: (1) whether neurofeedback-driven downregulation of taste focus reduces hedonic mindset; and (2) how neurofeedback affects pre/post craving measures and actual chocolate intake. This approach provides a mechanistic, neurobiologically grounded method for modifying food cue responsivity, addressing the limitations of previous cognitive and neurofeedback interventions.

\newpage

\section{Method \& Materials}

\subsection{Participants}
Thirty healthy adults (X women, X men) aged between X and X years (mean = XX.XX, SD = X.XX) were recruited through advertisements at Maastricht University. All participants met the eligibility criteria, including a healthy Body Mass Index (BMI; 18.5–25), a self-reported history of no mental or eating disorders, and a strong self-reported liking for chocolate (score above 75 on a 0–100 Visual Analogue Scale in response to "How much do you like chocolate?"). Participants also completed the Trait Chocolate Craving Questionnaire (TCCQ; \cite{bentonDevelopmentAttitudesChocolate1998}) to validate their self-reported love of chocolate. A TCCQ score was considered high if it exceeded 68.6 for women or 48.9 for men—thresholds corresponding to one standard deviation above the gender-specific normative means (M = 46.8, SD = 21.8 for women; M = 29.5, SD = 19.4 for men).

Participants were instructed to consume a small meal exactly two hours before the experimental session and to refrain from eating or drinking in the two hours leading up to the study. This procedure was necessary to standardise hunger across participants. All participants had normal or corrected-to-normal vision and no known neurological or psychological conditions. Late exclusion criteria also included MRI contraindications (e.g., electronic or ferromagnetic implants, a history of claustrophobia, or panic attacks).

After completing the eligibility screening, participants were directed to the Image Selection \& Rating Questionnaire, where they viewed a set of chocolate images from a local database (\textbf{\textit{database name to be confirmed}}). Participants ranked these images based on personal preference and selected their top 25, which were subsequently used as personalised stimuli in the experimental tasks. The study was approved by the Ethics Review Committee Psychology and Neuroscience of the University of Maastricht, and all participants provided written informed consent prior to participation.

\subsection{Screening Materials and Online Questionnaires}
Participants reported their age, BMI, and history of mental or eating disorders and were administered the TCCQ. This assessment covered various aspects, including difficulty controlling chocolate consumption, anticipation of a pleasant or negative experience, expected relief from negative mood, experience of craving, and concerns about weight or body image. Participants responded to each statement using a digital sliding scale ranging from 'Not at all like me' (0) to 'Very much like me' (100). The position of the slider indicates how strongly each statement reflected their feelings, with responses recorded on a continuous scale from 0 to 100.


\subsection{Stimuli Selection}
Participants who met the eligibility criteria were directed to the \textit{Image Selection \& Rating Questionnaire}, on qualtrics, where they were presented with a collection of chocolate images displayed simultaneously. These images were sourced from a local database (database name to be confirmed + add references), featuring a variety of chocolate types, such as dark, milk, and white chocolate, as well as different formats like chocolate bars, truffles, and chocolate-covered snacks.

Participants were instructed to rate each image based on how appealing and desirable they found the chocolate, using a digital sliding scale ranging from 'Not at all appealing' (0) to 'Very appealing' (100). They were informed that the results would rank the images based on their personal preferences and select the 25 most appealing ones. The chosen images were then used as personalised stimuli in subsequent experimental tasks. This ranking process allowed for the selection of chocolate images that best reflected each participant’s individual tastes, enhancing the ecological validity of the study.

\subsection{Procedure}
All sessions were recorded during the afternoon, as it has been shown that craving for the study stimuli is highest at this point in the day. Upon arrival, they completed a brief questionnaire assessing compliance with these instructions, current hunger levels, and chocolate craving for both milk and dark chocolate. Hunger was rated on a 0–100 Visual Analogue Scale (VAS), with anchors labelled “Not hungry at all” (0) and “Extremely hungry” (100). Chocolate craving was assessed separately for milk and dark chocolate using a digital VAS ranging from “I do not crave it at all” (0) to “Extremely appealing” (100). Neurofeedback training was conducted in a single session and consisted of one functional localiser scan and four runs of motivational neurofeedback training.

\paragraph{Functional localiser:}Participants performed an n-back task with two attentional focus conditions: hedonic and neutral, presented in an alternating manner. Each condition was introduced by a 1-s block cue, indicating whether participants should focus on tastiness (hedonic) or colour (neutral). Participants were then presented with two images sequentially, each for 1.5 s, followed by a 1.4-s response window, during which they indicated via button press which image was more or less colourful (neutral condition) or more or less tasty (hedonic condition). A total of 6 images per block was shown for a total of 17s. Within each block there were no repetitions of single pictures. Each condition was presented in 12 interleaved blocks, resulting in a total of 24 (12 per condition). The task was preceded by a 15-s fixation cross, and each block followed the same structure. Participants viewed the images via a back-projected screen, visible through a mirror mounted on the MRI head coil. All images were coloured and had a 1024 × 768 pixel resolution. Based on the statistical contrast between hedonic and neutral conditions in a whole-brain General Linear Model (GLM), we selected for each participant a target area showing reliable activation (top 10\% best voxels across all regions of interest) in the statistical maps derived from the localiser run.

\paragraph{Motivational neurofeedback - Swipe Down:} Participants were again presented with chocolate images and instructed to focus on taste perception. Unlike passive viewing tasks, participants were explicitly told that their goal was to mentally “swipe” the image downward, pushing it off the screen through their own neural regulation efforts. This design leverages motivational neurofeedback, enhancing ecological validity by directly linking neural control to a meaningful, goal-directed action.
Neurofeedback was provided intermittently rather than continuously to ensure a clear temporal separation between the regulation period and the feedback period. This approach prevents the movement of the image itself from triggering unintended brain activity, which could bias results. By isolating the participant’s voluntary downregulation efforts, we ensure that observed neural changes are attributable to self-regulation rather than visual motion cues. Instead of conventional feedback methods, such as a thermometer, the stimulus itself served as feedback, making the task more intuitive and engaging.
The participants were instructed to mentally engage in strategies to reduce the motivational salience of chocolate and “swipe” the image downward using mental imaging. This approach relies on intrinsic participant motivation, with the objective of reinforcing the connection between cognitive control and behavioural outcome.

\paragraph{Real time analysis:} Real-time preprocessing and analysis were performed using Turbo-BrainVoyager (Brain Innovation, Maastricht, The Netherlands) in combination with a custom Python script. During each neurofeedback block, the percent signal change (PSC) was calculated in real-time for each participant's region of interest (ROI) relative to a baseline referencial period.

The PSC at timepoint $t$, denoted as $\mathrm{PSC}_t$, was computed as:

\begin{equation}
\mathrm{PSC}_t = \frac{PSC_t - PSC_{\mathrm{ref}}}{PSC_{\mathrm{ref}}} \times 100
\end{equation}

where $PSC_t$ is the current BOLD signal intensity in the ROI and $PSC_{\mathrm{ref}}$ is the average signal during the preceding rest period. As such, $\mathrm{PSC}_t$ is expressed in percent units. To translate the PSC into visual feedback, the signal was first normalised relative to a participant-specific baseline value $\mathrm{PSC}_{\mathrm{ref}}$ and an expected signal variation range $\Delta_{\mathrm{PSC}}$. The normalised signal, corresponding to the neurofeedback provided to the participant ($fb$), was computed as:

\begin{equation}
fb = \left( \frac{\mathrm{PSC}_t - \mathrm{PSC}_{\mathrm{ref}} + \Delta_{\mathrm{PSC}}}{2 \Delta_{\mathrm{PSC}}} \right) \times 2 - 1
\end{equation}

This mapping transforms PSC values into the range $[-1, 1]$, with $x_t = 0$ corresponding to the reference level. The reference PSC ($\mathrm{PSC}_{\mathrm{ref}}$) was set per participant as the mean PSC during the hedonic attentional focus condition for the localiser run. This calculation utilised a built-in Turbo-BrainVoyager function that computes percent signal change per condition in real-time. The scaling factor $\Delta_{\mathrm{PSC}}$ was set to 0.5, reflecting an expected maximum deviation of $\pm 0.5$ PSC from reference baseline based on pilot data. %(this range could be adapted per participant based on signal characteristics observed during a calibration phase this week) 

To ensure a stable visual display, the continuous value $fb$ was discretised using 11 evenly spaced steps between $-1$ and $1$:

\begin{equation}
\hat{fb} = \underset{r \in \{-1.0, -0.8, \dots, 1.0\}}{\operatorname{argmin}} \left| x - fb \right|
\end{equation}

The discretised value $\hat{fb}$ was used to determine the vertical position of a visual feedback element (e.g., a food image), which moved smoothly in real time to reflect the participant’s success in modulating ROI activity. Downward movement ($r <0$) indicated successful downregulation of the targeted brain region.

\subsection{MRI Data Acquisition \& Neurofeedback Setup}

\paragraph{(f)MRI data acquisition} was performed at Scannexus (Ultra-High-Field MRI center, Maastricht, The Netherlands), using a 3T MRI scanner (3T MAGNETOM Prisma Fit, Siemens Medical Systems, Erlangen, Germany)with a 64-channel head coil. Participants lay comfortably in the scanner with their heads stabilised using foam pads on both sides. Stimuli were presented via a mirror attached to the head coil.
A high-resolution T1-weighted anatomical scan was acquired using an optimised magnetisation-prepared rapid gradient-echo (MPRAGE) sequence (TR = 2250 ms, TE = 2.21 ms, flip angle = 9°, FOV = 256 × 192 mm², voxel size = 1 × 1 × 1 mm). T2*-weighted functional images were acquired in an axial interleaved fashion using a gradient-echo echo planar imaging (EPI) sequence (TR = 1000 ms, TE = 31 ms, flip angle = 62°, FOV = 225 mm, voxel size = 2.5 × 2.5 × 2.5 mm, 48 slices). The EPI sequence was optimised to minimise distortion artefacts and susceptibility-related signal loss. The localiser run consisted of 783 brain volumes and each functional run consisted of 532 brain volumes; ensuring whole-brain coverage and high temporal resolution for neurofeedback and cognitive task-based analyses.

\paragraph{Real-time functional imaging} was performed by implementing a custom function in the MR image reconstruction pipeline which exported pixel data to an additional computer as soon as it became available. TurboExport (version 2.0, Brain Innovation, Maastricht, the Netherlands) was used to transform incoming pixel data for each volume into an image. Each resulting image was preprocessed in real time using Turbo-BrainVoyager (version 4.4; Brain Innovation, Maastricht, the Netherlands). The stimulation computer communicated with Turbo-BrainVoyager via a network connection, using the Transmission Control Protocol (TCP), in order to request the preprocessed real-time data to generate the feedback display.

\subsection{MRI data preprocessing and non-real time analysis}
The images were slice scan time-corrected using cubic-spline interpolation, 3D motion-corrected with sinc interpolation and a high-pass filter of three sines was applied. Motion parameters were inspected for quality control. The functional images were co-registered to the anatomical images. Deformation fields derived from segmentation were used to transform all the functional images to Talairach space. Lastly, we performed spatial smoothing with a Gaussian Kernel of 6 mm full width half-maximum (FWHM). Both the reorganisation scripts in MATLAB and the preprocessing Python scripts (utilised within the BrainVoyager Notebook) tool are publicly available on GitHub at https://github.com/I-ara?tab=repositories.

Analyses were performed in BrainVoyager. We employed a whole-brain analysis of variance (ANOVA) to investigate group-level differences. We applied a significance threshold of $p < 0.05$, and corrected voxel-wise for multiple comparisons using the Bonferroni method, to control for family-wise errors .

\section{Results}

\section{Discussion}

\section{Conclusion}

\section*{Data and Code Availability}

Preprocessing and analysis data scripts can be found at: https://github.com/I-ara?tab=repositories.

\section*{Author Contributions}

Iara de Almeida Ivo: Conceptualization; methodology; formal analysis; software; data curation; writing—original draft; visualisation; project administration.
Anne Roefs: Conceptualization; methodology; formal analysis; writing—original draft; writing—review and editing; supervision; project administration; funding acquisition. 
Bettina Sorger: Conceptualization; methodology; formal analysis; writing—review; supervision; project administration;
David Linden: Conceptualization; methodology; formal analysis; writing—review and editing; supervision; project administration;
Assunta Ciarlo: Sonceptualization; software.

\section*{Funding}

This study is part of the project “New Science of Mental Disorders” (www.nsmd.eu), supported by the Dutch Research Council and the Dutch Ministry of Education, Culture and Science (NWO gravitation grant number 024.004.016). Anne Roefs was supported by a VIDI-grant (452.16.007) of the Netherlands Organization for Scientific Research.

\section*{Declaration of Competing Interests}

The authors declare no conflicts of interest.

%\section*{Acknowledgements}
%
%Acknowledgements text (optional).

\section*{Supplementary Material}

Supplementary Material (created during production as a web link to online material).

\printbibliography

\appendix

\section{Appendix}

Appendices (optional).


\end{document}
